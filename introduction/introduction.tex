\chapter{Introduction}

% Setting out the aims and objectives of your project, explaining the overall intention of the project and specific steps that will be taken to achieve that intention.

\section{Motivation}

A 2015 study found that smartphone users spend an average of 32 minutes per day on WhatsApp \cite{montag2015}, highlighting the fact that in modern times, social media and instant messaging applications are considered by many to be one of the most convenient and favoured methods of communication \cite{church2013}. Users enjoy the simplicity and ease of use of the small group communication that these applications provide, however relying on these social media services for communication also presents a number of privacy concerns. For example; many users are unaware that messages sent over Facebook do not have end-to-end encryption by default, and that this feature must be manually enabled for individual conversations \cite{facebook2017}. The result of this is that Facebook has the ability to analyse the content of these unencrypted messages and use the data that it collects in whatever way it deems fit, such as harvesting data on users in order to target political advertising to them, a practice that was revealed as part of the Facebook-Cambridge Analytica scandal in 2018 \cite{guardian2018}.

In addition, even in applications which do claim to use end-to-end encryption, such as WhatsApp \cite{whatsapp2017}, if these are closed-source (as is the case for WhatsApp), it is impossible for users to truly verify the extent with which their personal data is protected. It is widely reported that WhatsApp uses the Signal protocol for end-to-end encryption \cite{whatsapp2017}, however WhatsApp users must trust that their private keys are not sent to the WhatsApp servers, which would allow WhatsApp (and its parent company Facebook) to decrypt and read the messages.

Despite the continued growth of social media services, email is still relied upon as the backstop communication method. Creating a social media account requires users to have an email address, and as such, social media users can be considered as a subset of email users. Since email already provides a communication method between people, it is seemingly unnecessary for people to sign up to third party communication providers such as WhatsApp, with its inherent associated risks as previously discussed. However, there are not currently any alternatives which increase the usability of email and retain its federated benefits. In addition, email is not encrypted by default, and current solutions to this are complex to set up and use, which means that they see little adoption.

Aside from the fact that there were 3.9 billion email users in 2019 \cite{radicati2019}; representing over 50\% of the world's population, using email protocols and addresses as a means for communication has a number of advantages \cite{hanson2011}. Email is by design, the ultimate federated service, with email servers around the world working both independently and together to achieve the goal of delivering messages. In addition, email inherently supports self-hosting of mail servers, giving total control to users who require this - removing any need for reliance or trust of third-party providers.

Unfortunately, email clients have failed to keep pace with our reliance on email, and most have seen few major changes since their inception \cite{rohall2004}. A problem with email in its current form is that it does not effectively handle group communication with explicit message threading in the same way that social media platforms do which makes conversations with numerous participants difficult to follow. The mismatch between the user interfaces for email clients and users' needs for handling email has been extensively documented, and one recurrent theme in work to improve the experience is that messages should appear as explicit conversations rather than existing independently \cite{venolia2003}.

Due to the underlying way in which emails are currently sent and received, simply changing the user interface is not enough in order to truly improve the user experience. Therefore, a system that combines the power of email as a federated transport mechanism with the usability of social media messaging appears to be worthy of further investigation and implementation with the intention of solving some of the issues presented above. 

\section{Project Aim and Objectives}

The aim of this project is to design and develop an application that enables small, closed-group communication, using email as the underlying transport mechanism for messages, to mitigate the privacy concerns that exist in existing communication applications that use centralised architectural models.

The key objectives that have been identified as necessary steps to achieve this aim are as follows:

\begin{enumerate}
  \item \label{itm:data-structure} Investigate and design a suitable data structure for storing and transmitting message threads.
  \item Investigate and implement a suitable architectural model for the system such that the federated aspect is preserved.
  \item Design and build an intuitive user interface for the messaging system.
  \item Implement the application logic for sending and receiving messages utilising the data structure designed in Objective \ref{itm:data-structure}.
  \item Investigate and implement a suitable method of securing the messages in transit, to ensure that messages can only be read by the intended recipients.
  \item Investigate and implement a method of allowing files, such as images, to be sent in message threads securely and efficiently.
\end{enumerate}