\chapter{Related Work}

% Explaining what your project does that is new or is better than existing work in the same field.

There exists a large number of commercial applications on the market today that allow small group messaging, with explicit threading built-in. Some examples, which are referred to throughout this report, include WhatsApp, Facebook Messenger, Slack, and Telegram. Many of these services provide a much richer feature-set than can be realised using email alone with current clients. However, none of these applications provide the federated, decentralised model that is such a benefit of email, instead each relying on their own servers for transport and delivery.

One example of an attempt to build an email client that solves some of the core problems with email, including ``lack of context'' due to inadequate message threading systems, is the ReMail project from IBM research \cite{kerr2004designing}. Whilst this project does attempt to improve the email reading experience, it leaves the way in which email is used fundamentally unchanged. The result of this is a client that fundamentally still works like, and has all the quirks of traditional email communication, rather than the traits of modern messaging systems that are increasingly becoming the preferred way for people to communicate, especially within groups.

Rather than simply creating a new client interface for reading email, this project aims to ignore the current way in which emails are constructed. Instead, it will simply use email as the ultimate federated network of users, since virtually everybody has an email account, to act as the transport mechanism for messages being sent using a newly designed schema.